\section{Reading 1: Diaz et al 2019}

\subsection{Summary}

\begin{itemize}
	\item \textbf{telecoupling} -- human actions incresingly act at a
		distance due to globalization
	\item increasing demand between supply and demand due to global trade
	\item the analysis pinpoints five crucial levers (priority
		interventions) and eight leverage points (for intervention)
\end{itemize}

\subsection{Intro}

\begin{itemize}
	\item the human impact on life on Earth has increased since the 1970s
	\item both the benefits of economy and the cost of reducing nature
		are unequally distributed
\end{itemize}

\subsection{Taking stock of the fabric of life}

\begin{itemize}
	\item over the past 50 years, the quality of nature to support life
		has declined on 14 of 18 categories identified by IPBES
		\footnote{Intergovernmental Platform on Biodiversity and
		Ecosystem Services}
	\item exceptions to the downward trend are: regulation of ocean
		acidification, energy, food and feed, materials and assistance
	\item more than 800 million people still face chronic food deprivation
	\item the biomass of world's vegetation has halved over human history
	\item forest area is only 68\% of its preindustrial size
\end{itemize}

\subsection{Direct and indirect drivers of change}


\begin{itemize}
	\item Direct drivers:
		\begin{itemize}
			\item land/sea use change
			\item direct exploitation
			\item climate change
			\item pollution
			\item invasive alien species
			\item others
		\end{itemize}
	\item Indirect drivers:
		\begin{itemize}
			\item demographic and sociocultural
			\item economic and technological
			\item institutions and governance
			\item conflicts and epidemics
		\end{itemize}
	\item examples of declines in nature:
		\begin{itemize}
			\item natural ecosystems have declined by 47\%
				on average, relative to their previous states
			\item 25\% of species are already threatened by
				extinction
			\item biotic integrity -- abundance of naturally
				present species -- has declined by 23\% on
				average in terrestial communities
			\item the global biomass of wild mammals has fallen by
				82\% 
			\item 72\% of indicators developed by Indigenous
				Peoples and local communities have deteriorated
		\end{itemize}
\end{itemize}

\subsection{Progress towards internationally agreed goals}

\begin{itemize}
	\item 20 Aichi Targets in the Strategic Plan on Biodiversity
		2011-2020
		\begin{itemize}
			\item contains 54 elements
			\item good progress has been made on 5
			\item moderate progress towards 19
			\item poor progress or movement away from target on 21
			\item unknown progress on 17 elements
		\end{itemize}
	\item 
\end{itemize}

\subsection{Levers and leverage points for transformative change}

Levers:

\begin{itemize}
	\item incentives and capacity building
	\item cross-sectional cooperation
	\item pre-emptive action
	\item decision-making in the context of resilience and uncertainty
	\item environmental law and implementation
\end{itemize}

Leverage points:
\begin{itemize}
	\item embrace diverse vision of a good life
	\item reduce total consumption and waste
	\item unleash values and action
	\item reduce inequalities
	\item practice justice and inclusion in conservation
	\item internalize externalities and telecouplings
	\item ensure environmentally friendly technology, innovation and
		investment
	\item promote education and knowledge generation and sharing
\end{itemize}
