\section{Lecture 1}

\subsection{Housekeeping}

\begin{itemize}
	\item Every Tuesday
	\item 18 - ca. 20h
	\item Live or recorded on Zoom
	\item All lectures will be recorded and available on Canvas
	\item Two in-person meetings
	\item Gothenurg Natural History Museum
	\item Gothenburg Botanical Garden
	\item Assignment: develop and submit an essay on a topic we cover in
		the course that is of particular personal interest. Open-ended
		and offers the opportunity for personal reflection, while also
		gaining experience in more scientific-styled writing.
\end{itemize}

\subsection{Biodiversity bias}

\begin{itemize}
	\item 2 million -- 1 trillion species on Earth
	\item ca. 1.7 million databased
	\item Knowledge derives from charismatic, macroscopic species.
\end{itemize}

\subsection{Mechanisms of biodiversity formation}

\begin{itemize}
	\item speciation
	\item extinction
	\item migration
\end{itemize}

Adaptation = the process which enables organisms to adjust to their
environment in order to ensure survival. This can lead to speciation.

Example: polar bears have extremely large feet to distribute weight better
on snow.

\subsection{Speciation}

Types of speciation:
\begin{itemize}
	\item allopatric -- populations get geographically distanced which
		limites the gene flow
	\item peripatric -- a small group gets isolated
	\item parapatric -- populations are adjacent but not completely
		sepatated
	\item sympatric -- species evolve from a single ancestral species
		while living in the same area
\end{itemize}

\subsection{Extinction}

The termination of any organism.

Species have a lifespan, on average 2 million years. After that, species either
speciate or go extinct.

\subsection{Dispersal}

Dispersal = movement of individuals (through geological time, such as
marsupials colonizing North America once South and North Americas got connected
through geological time)

Since the industrial revolution, many species are progressively changing their
distribution and dispersing further north.

\subsection{Latitudinal species gradient}

Closer to equator/tropics = more species (also more human languages!)

Does the diversity in species and language covary?

\subsection{Species}

How do we measure species for example in tropical jungle with countless
species?

Quandrants: count within a small area then you can extrapolate

Species richness = the number of different species in an area

Numerical species richness = number of species per specific number of
individuals

Species density = number of species per unit area

Species abundance = the relative abundance of species

Amazonia is known to be highly biodiverse, but a few species are very abundant
and most species are incredibly rare. 1\% of species make up for 50\% of all
trees.

\subsection{Learning outcomes}
\begin{itemize}
	\item Biodiversity: number of differential biological variants found in
		a given place and time
	\item Speciation, extinction, and dispersal build biodiversity
	\item Biodiversity is not evenly distributed and can be measured in
		different ways
\end{itemize}
